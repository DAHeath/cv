\documentclass[11pt,a4paper,sans]{moderncv}

\usepackage{verbatim}
\usepackage{textcomp}

% moderncv themes
\moderncvstyle{classic}
\moderncvcolor{blue}
\renewcommand*{\namefont}{\fontsize{38}{40}\mdseries\upshape}

\usepackage[utf8]{inputenc}

% adjust the page margins
\usepackage[scale=0.8]{geometry}

\firstname{David}
\familyname{Heath}
\address{1101 Juniper Street NE, Unit 211}{Atlanta, Georgia, 30309}
\mobile{770-361-6450}
\email{heath.davidanthony@gatech.edu}

\begin{document}
\maketitle

\section{Education}
\cventry{2016-Present}{PhD, Computer Science}{Georgia Institute of Technology}{Atlanta, Georgia}{}{}
\cvline{advisor:}{\small Vlad Kolesnikov}
\cvline{GPA:}{\small 4.0}
\cventry{2010-2014}{BS, Computer Science}{}{}{}{}
\cventry{}{BS, Mechanical Engineering}{Georgia Institute of Technology}{Atlanta, Georgia}{}{}
% \cvline{overall gpa:}{\small 3.79/4.0}
\cvline{honors:}{\small Summa Cum Laude}
\cventry{Summer 2012}{Study Abroad Program}{Georgia Tech Lorraine}{Metz, France}{}{}

\section{Publications}

\cventry{2020}{
D. Heath and V. Kolesnikov.
"Stacked Garbling for Disjunctive Zero-Knowledge Proofs".
  EUROCRYPT 2020
}{%
}{%
}{%
}{%
}

\cventry{2019}{Q. Zhou, D. Heath, and W. Harris. Relational
  Verification via Invariant-Guided Synchronization. HCVS 2019}{}{}{}{%
}
\cventry{2018}{Q. Zhou, D. Heath, and W. Harris. Solving Constrained
Horn Clauses Using Dependence-Disjoint Expansions. HCVS 2018}{}{}{}{%
  Constrained Horn Clauses (CHCs) are a class of logic programming problem
  used to verify software. We developed a
  CHC solver that surpassed the state of the art in both memory and time usage.
}
% \cventry{2017}{D. Heath, H. Kwon, W. Harris, and H. Esmaeilzadeh.
% Trustable, Locked, Reconfigurable Logic. TECHCON 2017}{}{}{}{%
%   `Logic locking' prevents the leakage of
%   circuit designs. We developed
%   and presented techniques for proving properties of circuits which
%   have undergone logic locking.
% }

%
% \cventry{2016}{D. Heath and T. Stittleburg. Data Flow Oriented
% Software Design in FACE\texttrademark. FACE\texttrademark~Technical
% Interchange Meeting 2016}{}{}{}{%
%   The Future Airborne Capability Environment (FACE\texttrademark) is
%   an Open Group technical standard aimed at improving the quality of
%   avionics software. We developed and presented novel software
%   engineering techniques for implementing software applications in the
%   context of the standard.
% }

\section{Experience}

\cventry{2014-2016}{Research Engineer I}{Georgia Tech Research Institute}{Atlanta, GA}{}{
As part of GTRI's Electronic Systems Laboratory, I worked with
safety-critical software systems and helped to verify the Future
Airborne Capability Environment (FACE\texttrademark) Technical
Standard.
}

% \cventry{2013}{Undergraduate Assistant}{Georgia Tech Research Institute}{Atlanta, GA}{}{
% As an undergraduate assistant at GTRI, I was solely responsible for writing,
% maintaining, and documenting several large, in-house programming projects in C\#
% and C++ programming languages. Projects included data conversion applications
% among human readable formats and domain specific formats.
% }

\section{Awards}
\cventry{2016-2020}{Georgia Tech President's Fellowship}{}{}{}{%
  Georgia Tech offers a fellowship to the top 10 percent of PhD applicants.
}
\cventry{2017}{CS 7001 Research Project Award}{}{}{}{%
  Every Computer Science PhD student at Georgia Tech is required to
  take CS 7001, an introductory course to academic research. Each
  student is required to write and present work featuring research
  tasks conducted during the semester. I was presented an award for
  best research project as part of the Georgia Tech College of
  Comnputing Annual Awards and Honors ceremony.
}

% \section{Projects}

% \cventry{2014}{Computer Science Senior Design}{Hanon: Music Study Assistant}{}{}{
% For my senior design project I collaborated with two others students to write
% Hanon. Hanon is a prototype music study assistant which uses frequency analysis
% to determine when the musician has made mistakes.
% }

% \cventry{2014}{Mechanical Engineering Senior Design}{Bike Share Helmet Distribution System}{}{}{
% I collaborated with five other engineering students in a senior design project
% that involved designing and prototyping a bicycle helmet distribution kiosk for
% rentable Bike Share systems. Experiences included communicating with
% representatives of the Atlanta Bike Share coalition, ideation and concept
% generation, engineering analysis, and prototyping. Prototype was working model
% with several sensors and actuators, controlled by an Arduino.
% }

% \cventry{2013}{Alphabet Energy Design Competition}{Thermoelectric Phone Charger}{}{}{
% My collaborator and I were awarded \$1,000 for the design of a thermoelectric
% USB phone charger. Together, we researched and designed a realistic device that
% could be used for long-haul truck drivers or campers in case of an emergency.
% }

\end{document}
